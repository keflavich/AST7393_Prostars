\documentclass[twoside]{tufte-book} % Use the tufte-book class which in turn uses the tufte-common class
\pdfoutput=1
\usepackage{natbib}
\usepackage{xspace}

\usepackage{microtype} % Improves character and word spacing

\usepackage{lipsum} % Inserts dummy text

\usepackage{booktabs} % Better horizontal rules in tables
\usepackage{epsdice}

\usepackage{graphicx} % Needed to insert images into the document
\graphicspath{{graphics/}} % Sets the default location of pictures
\setkeys{Gin}{width=\linewidth,totalheight=\textheight,keepaspectratio} % Improves figure scaling

\usepackage{fancyvrb} % Allows customization of verbatim environments
\fvset{fontsize=\normalsize} % The font size of all verbatim text can be changed here

\newcommand{\hangp}[1]{\makebox[0pt][r]{(}#1\makebox[0pt][l]{)}} % New command to create parentheses around text in tables which take up no horizontal space - this improves column spacing
\newcommand{\hangstar}{\makebox[0pt][l]{*}} % New command to create asterisks in tables which take up no horizontal space - this improves column spacing

\usepackage{xspace} % Used for printing a trailing space better than using a tilde (~) using the \xspace command
\newcommand{\veco}{\ensuremath{\mathbf{\Omega}}\xspace}
\newcommand{\vecv}{\mathbf{v}\xspace}
\newcommand{\msun}{\ensuremath{\mathrm{M}_\odot}\xspace}
\newcommand{\lsun}{\ensuremath{\mathrm{L}_\odot}\xspace}
\newcommand{\ehat}{\hat{\mathbf{e}}\xspace}
\usepackage{amsmath}
\usepackage{mathtools}


\begin{document}

\noindent\textbf{Problem Set 0: Molecular Clouds}

``Typical'' molecular clouds have mean densities of $\sim100$ H$_2$ molecules per
cubic centimeter and span scales $\sim10$ pc.

The mean molecular mass is 2.8 atomic mass units (AMU or Daltons) per H$_2$
molecule (which is equivalent to 2.3 AMU per free particle - Helium atoms are
also free particles).

Molecular clouds have temperature $T\sim20$ K.
The sound speed in molecular gas is $c_s = \sqrt{k_B T}{m}$, where $m$ is the
mass of the particle, $k_B$ is the Boltzmann constant, and $T$ is the
temperature in Kelvin.

\begin{enumerate}
    \item Compute: What is the free-fall timescale in molecular gas?
    \item Compute: What is the crossing time of a typical molecular cloud?  The
        crossing time is the time required for a sound wave to cross the cloud.
    \item How do these compare?
    \item If the molecular cloud forms stars at 100\% efficiency (all of the
        gas becomes stars at some point), and it collapses in one free-fall
        time, what is the star formation rate?
        Do we expect stars to form at this rate?
\end{enumerate}

\end{document}


