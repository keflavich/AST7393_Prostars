\documentclass{article}
\usepackage[inner=2.5cm,outer=2.5cm,top=4.5cm,bottom=4.5cm]{geometry}
\pdfoutput=1
%\usepackage{xspace} % Used for printing a trailing space better than using a tilde (~) using the \xspace command

\begin{document}

\title{Project for AST 4930: Star Formation}
\noindent \textbf{\large Project for AST 4930: Star Formation}

You will do a guided project addressing one of the key topics in the
class.

This will serve as training both for research and teaching.
It will give you an opportunity to practice formal speaking.

You will do the following:
\begin{enumerate}
    \item Identify a topic relevant to the course by picking from one of the review articles.
    \item Read the review.  
    \item Prepare a presentation on the topic, summarizing the context of the
        problem, the problem, past and recent progress on the problem, and open
        questions related to the problem.
    \item Create new material based on the problem.  Either:
    \begin{itemize}
        \item Design an experiment, observation, or theoretical project that can
            address one of the open problems in the topic.  
        \item Prepare a problem set.  The problem set should
            include a quantitative demonstration of the core problem of the topic.
    \end{itemize}
\end{enumerate}



\noindent \textbf{Reading:}
You will read your selected article before Spring Break.

You will write a 1-page summary of the topic.  

You will include a maximum 1-page list of questions you have about the topic
(if any) and open questions on the topic.

\textit{Learning Goals}: Practice reading scientific literature and developing
your own questions about the literature.

\textit{Grading Rubric}:
\begin{itemize}
    \item 50\% was the material appropriately summarized?
    \item 25\% were reasonable open questions identified?
    \item 25\% technical: Is the formatting reasonable?  Is the English clear?
\end{itemize}

\noindent \textbf{Presentation:}

You will give a brief, recorded presentation on the selected topic. 
Your target audience is your classmates: i.e., students who are taking 4000-level astronomy classes.

Your presentation will be limited to 8 minutes.

You will present on the aspect of the review you found most exciting.
Your talk should:

\begin{enumerate}
    \item Briefly summarize the subtopic
    \item Describe why it is interesting
    \item Note what theoretical work has been done
    \item Note what observational work has been done
    \item Summarize the open questions as described in the review
    \item If the review is older than 10 years, briefly note the
    most exciting progress since the review
\end{enumerate}

Your presentation should include slides.
It should include figures from the review paper \emph{and} supplementary figures
from additional literature search.
For example, if you present on molecular clouds, you should show some
additional images of molecular clouds, not just those in the review article, to
illustrate the variety of cloud shapes that occur.

You will record a talk that is $<=$ 8 minutes long and turn it in in .avi or .mp4
format.  Editing is allowed.

\textit{Learning Goals}:
    Practice presentation skills and demonstrate communication ability.

    Practice video recording and/or editing skills.

\textit{Grading Rubric}:
\begin{itemize}
    \item 50\% was the material correctly and completely discussed?
    \item 25\% was the presentation clear and informative?
    \item 25\% completion
\end{itemize}

\noindent \textbf{Presentation Review:}

During finals period, you will watch your classmates' presentations.

For each presentation, you will turn in the following:
\begin{itemize}
    \item One or more sentences summarizing the presented material.
    \item A question about the material.  This can be something you didn't
    understand, or a new idea raised by what you learned.
\end{itemize}

\textit{Grading Rubric}:
Each presentation is worth 5 points.
The grade is 3 points for the summary, two for the question.

\noindent \textbf{(option) Problem Set:}

You will design a quantitative physics question based on your selected topic.

The physics question should resemble problem sets you've seen in classes.
The most straightforward questions to develop are more closely guided questions that are meant to
guide students to a common solution path.
It may be easier to ask an open-ended question, but it is more difficult to shape
such a question into a closed-form, gradeable question.
%problems with some simple, open-ended questions that require both insight and
%creativity to answer, and some more closely guided questions that are meant to
%guide students to a common solution path.

You only need 1 question on the topic.  
You will need to present both the problem and the solution.

\textit{Learning Goals}:
    Develop teaching skills.  Practice assessing which problems are tractable.

\textit{Grading Rubric}:
\begin{itemize}
    \item 33\% was a reasonably tractable problem identified?
    \item 33\% was the problem written out in a consistent, understandable way?
    \item 33\% was the solution obtained and presented?
\end{itemize}

\noindent \textbf{(option) Experiment Design:}

You will practice converting an open-ended, research question into an 
experiment or theory project.
You do not need to carry out the project, but you should outline it in
detail.
You will write up your experimental plan as an observing, computing time,
or funding proposal.

The proposal will be limited to 4 pages, including figures and references.
However, if you choose to write an observing, grant, or computing time proposal
for submission, you may use their format.

\textit{Learning Goals}:
    Practice experimental design and asking answerable scientific questions.

\textit{Grading Rubric}:
\begin{itemize}
    \item 25\% was an interesting problem identified?
    \item 25\% is a practical experiment described?
    \item 50\% does the written proposal compellingly describe the importance of the problem and the way it will be addressed?
\end{itemize}


\clearpage

\noindent \textbf{Available Topics:}
\begin{itemize}
    \item First Stars: 2020SSRv..216...48H, 2004ARA\&A..42...79B, https://www.annualreviews.org/content/journals/10.1146/annurev-astro-071221-053453
    \item Multiple Stellar Populations: 2022Univ....8..359M, https://www.annualreviews.org/doi/abs/10.1146/annurev-astro-081817-051839
    \item The Solar Neighborhood in the Age of Gaia
    \item OB associations: http://ppvii.org/chapter/04/, 2022arXiv221200067Z
    \item Magnetic Fields in Star Formation: From Clouds to Cores: 2022arXiv220311179P
    \item Magnetic Fields in Disks \& Outflows: 2022arXiv220913765T
    \item Accretion Variability: 2022arXiv220311257F
    \item Organic Chemistry in the First Phases of Solar-Like Protostars: 2022arXiv220613270C
    \item Structured Distributions of Gas and Solids in Protoplanetary Disks: 2022arXiv221013314B,\\ http://ppvii.org/chapter/12/
    \item Hydro-, Magnetohydro-, and Dust-Gas Dynamics of Protoplanetary Disks: 2022arXiv220309821L
    \item Measurements and Implications of the Fundamental Disk Properties: 2022arXiv220309818M
    \item Demographics of Young Stars and Their Protoplanetary Disks: 2022arXiv220309930M
    \item The Role of Disk Winds in the Evolution and Dispersal of Protoplanetary Disks: 2022arXiv220310068P
    \item Kinematic Structures in Planet-Forming Disks: 2022arXiv220309528P
    \item Planet-Disk Interactions and Orbital Evolution: 2022arXiv220309595P
    \item Planet Formation Theory - from pebbles to planets: 2022arXiv220309759D
    \item Short-Lived Radionuclides in Meteorites and the Sun’s Birth Environment: 2022arXiv220311169D
    \item Chemical Habitability: 2022arXiv220310056K
    \item Isotopic Links from Planet Forming Regions to the Solar System: 2022arXiv220310863N
    \item Astrochemistry: https://www.annualreviews.org/doi/abs/10.1146/annurev-astro-032620-021927
    \item Cometary Chemistry: https://www.annualreviews.org/doi/abs/10.1146/annurev-astro-091918-104409
    \item Star Clusters: https://www.annualreviews.org/doi/abs/10.1146/annurev-astro-091918-104430
    \item High-mass Star Formation: https://www.annualreviews.org/doi/abs/10.1146/annurev-astro-091916-055235
    \item Very Massive Stars: https://publishup.uni-potsdam.de/opus4-ubp/files/41522/pmnr601.pdf,  https://arxiv.org/abs/1406.4836
    \item Origin of the Solar System: https://arxiv.org/abs/2301.05212
    \item Planet formation theory: an overview: https://arxiv.org/abs/2412.11064
    \item Molecular Gas and the Star-Formation Process on Cloud Scales in Nearby Galaxies
    https://www.annualreviews.org/content/journals/10.1146/annurev-astro-071221-052651
    \item An Observational View of Structure in Protostellar Systems
     https://www.annualreviews.org/content/journals/10.1146/annurev-astro-052920-103752
     \item Dust Growth and Evolution in Protoplanetary Disks
     https://www.annualreviews.org/content/journals/10.1146/annurev-astro-071221-052705
     \item Protoplanetary Disk Chemistry
     https://www.annualreviews.org/content/journals/10.1146/annurev-astro-022823-040820
\item Star Formation in the Milky Way and Nearby Galaxies
 https://www.annualreviews.org/content/journals/10.1146/annurev-astro-081811-125610
    \item Observations of Protoplanetary Disk Structures  https://www.annualreviews.org/doi/abs/10.1146/annurev-astro-031220-010302
    \item Accretion onto Pre-Main-Sequence Stars https://www.annualreviews.org/doi/abs/10.1146/annurev-astro-081915-023347
    \item Formation of Globular Clusters https://arxiv.org/abs/2501.16438
    \item Low vs High mass star formation https://arxiv.org/abs/2501.16866
\end{itemize}





\end{document}
