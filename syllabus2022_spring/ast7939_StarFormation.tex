\documentclass[11pt]{article}
%\usepackage{geometry}
\usepackage[inner=1.5cm,outer=1.5cm,top=2.5cm,bottom=2.5cm]{geometry}
\pagestyle{empty}
\usepackage{graphicx}
\usepackage{fancyhdr, lastpage, bbding, pmboxdraw}
\usepackage[usenames,dvipsnames]{color}
\definecolor{darkblue}{rgb}{0,0,.6}
\definecolor{darkred}{rgb}{.7,0,0}
\definecolor{darkgreen}{rgb}{0,.6,0}
\definecolor{red}{rgb}{.98,0,0}
\usepackage[colorlinks,pagebackref,pdfusetitle,urlcolor=darkblue,citecolor=darkblue,linkcolor=darkred,bookmarksnumbered,plainpages=false,pdflang=en-US,pdftitle=AST3722_syllabus_fall2021]{hyperref}
\renewcommand{\thefootnote}{\fnsymbol{footnote}}

\pagestyle{fancyplain}
\fancyhf{}
\lhead{ \fancyplain{}{AST7939} }
%\chead{ \fancyplain{}{} }
\rhead{ \fancyplain{}{\today} }
%\rfoot{\fancyplain{}{page \thepage\ of \pageref{LastPage}}}
\fancyfoot[RO, LE] {page \thepage\ of \pageref{LastPage} }
\thispagestyle{plain}

%%%%%%%%%%%% LISTING %%%
\usepackage{listings}
\usepackage{caption}
\DeclareCaptionFont{white}{\color{white}}
\DeclareCaptionFormat{listing}{\colorbox{gray}{\parbox{\textwidth}{#1#2#3}}}
\captionsetup[lstlisting]{format=listing,labelfont=white,textfont=white}
\usepackage{verbatim} % used to display code
\usepackage{fancyvrb}
\usepackage{acronym}
\usepackage{amsthm}
\VerbatimFootnotes % Required, otherwise verbatim does not work in footnotes!



\definecolor{OliveGreen}{cmyk}{0.64,0,0.95,0.40}
\definecolor{CadetBlue}{cmyk}{0.62,0.57,0.23,0}
\definecolor{lightlightgray}{gray}{0.93}



\lstset{
%language=bash,                          % Code langugage
basicstyle=\ttfamily,                   % Code font, Examples: \footnotesize, \ttfamily
keywordstyle=\color{OliveGreen},        % Keywords font ('*' = uppercase)
commentstyle=\color{gray},              % Comments font
numbers=left,                           % Line nums position
numberstyle=\tiny,                      % Line-numbers fonts
stepnumber=1,                           % Step between two line-numbers
numbersep=5pt,                          % How far are line-numbers from code
backgroundcolor=\color{lightlightgray}, % Choose background color
frame=none,                             % A frame around the code
tabsize=2,                              % Default tab size
captionpos=t,                           % Caption-position = bottom
breaklines=true,                        % Automatic line breaking?
breakatwhitespace=false,                % Automatic breaks only at whitespace?
showspaces=false,                       % Dont make spaces visible
showtabs=false,                         % Dont make tabls visible
columns=flexible,                       % Column format
morekeywords={__global__, __device__},  % CUDA specific keywords
}

\title{AST 7939: Star Formation, Spring 2022}

%%%%%%%%%%%%%%%%%%%%%%%%%%%%%%%%%%%%
\begin{document}
\begin{center}
{\Large \textsc{AST 7939: Star Formation}}
\end{center}
\begin{center}
Spring 2021
\end{center}
%\date{September 26, 2014}

\begin{center}
\rule{6in}{0.4pt}

\begin{minipage}[t]{.95\textwidth}
\begin{tabular}{llccll}
\textbf{Instructor:} & Prof. Adam Ginsburg &  &  & \textbf{Time:} & M,W,F | Period 7 (1:55 PM - 2:45 PM) \\
\textbf{Email:} &  \href{mailto:adamginsburg@ufl.edu}{adamginsburg@ufl.edu} & &  & \textbf{Place:} & Rm 3, Bryant Space Science Center,\\
 &&&&&  and Zoom\\
  &&&& \textbf{Office Hours:} &  by appointment\\
  \hline \\
\textbf{TA:} & TBD & &  & \textbf{Office Hours:} & TBD, or by appointment    \\
\textbf{Email:} &  \href{mailto:}{} &&&&  \\
&&&&& \\
\end{tabular}
\end{minipage}

\rule{6in}{0.4pt}
\end{center}
\vspace{.5cm}
\setlength{\unitlength}{1in}
\renewcommand{\arraystretch}{2}

\textbf{Timing / Structure:}\\
The course will be taught primarily in person.
Class will consist of a mixture of lectures and homework.
The course will be taught largely as a classic lecture course so that we can cover a wide range of material.
Active learning will be employed when possible, but since this is my first time teaching the course,
I may be limited in how much active learning I can incorporate.



\vskip.15in
\noindent\textbf{Course Pages:} \begin{enumerate}
    \item \url{https://github.com/keflavich/AST7393_Prostars}
    \item \url{https://ufl.instructure.com/courses/444877}
\end{enumerate}


\vskip.15in
\noindent\textbf{Communication:}\\
Communication will be via Canvas, Zoom, and Slack.

\vskip.15in
\noindent\textbf{Office Hours:}\\
Office hours will be primarily virtual and held via Slack, with escalation to Zoom as needed.

\vskip.15in
\noindent\textbf{Main References:} %\footnotemark
%This is a  restricted list of various interesting and useful books that will be touched during the course. You need to consult them occasionally.
\begin{itemize}
    \item Star Formation Notes by Mark Krumholz \url{https://github.com/Open-Astrophysics-Bookshelf/star_formation_notes}
    \item The Formation of Stars by Steven Stahler and Francesco Palla \url{https://ui.adsabs.harvard.edu/abs/2004fost.book.....S/abstract}
    \item Accretion Processes in Star Formation by Lee Hartmann \url{https://ui.adsabs.harvard.edu/abs/1998apsf.book.....H/abstract}
    \item Tom Megeath's ``The Secret Lives of Stars'' \url{http://astro1.physics.utoledo.edu/~megeath/ph6820/ph6820.html}

\end{itemize}

% \footnotetext{Downloadable ebook versions are available on AeLP.}

\vskip.15in
\noindent\textbf{Objectives:}
You will learn the fundamentals of star formation, from gravitationl collapse
of a molecular cloud to formation of a nuclear burning core.

You will gain experience solving physical problems and developing physical insights.
You will learn about physical processes and order-of-magnitude estimation techniques.

This class will fill in the size scales between ISM and stars, and will have some overlap with each.

%\vskip.15in
%\noindent\textbf{Prerequisites:}



\vspace*{.15in}

\noindent \textbf{Course Outline:}
\begin{center}
\begin{minipage}{5in}
\begin{flushleft}
%Chapter 1 \dotfill ~$\approx$ 3 days \\
    Learning goals of the course:
    \begin{enumerate}
        \item Understand the fundamentals of star formation theory and observables
        \item Practice formulating, and answering, questions for research
        \item Learn key concepts and practice working with them to develop a vocabulary for discussing star formation
    \end{enumerate}
\end{flushleft}
\end{minipage}
\end{center}


\vspace*{.15in}
\noindent\textbf{Grading Policy:}
\begin{itemize}
    %\item Labs (30\%)
    %\item Final Assignment (20\%)
    \item Class Assignments and participation, homework (60\%)
    \item Project (30\%)
    \item Exams (10\%)*
    %\item Final Presentation (10\%)
\end{itemize}

The late policy is 10\% credit lost per day.  However, I generally will give extensions if late assignments are well-justified and excused in advance.
No credit will be given for the final project if it is late.

*: I do not guarantee that there will be exams; if there are, they will comprise 10\% of the grade, otherwise, grades will be based only on projects
and class and homework.

More information on UF grading policies is here:
\url{https://gradcatalog.ufl.edu/graduate/regulations/#text}


Letter grades are: \\
\begin{tabular}{|c|c|}
    \hline
    Letter & Minimum \% \\
    \hline
    A & 93 \\
    A- & 90 \\
    B+ & 87 \\
    B & 84 \\
    B- & 80 \\
    C+ & 77 \\
    C & 74 \\
    C- & 70 \\
    D+ & 67 \\
    D & 64 \\
    D- & 60 \\
    \hline
\end{tabular}

I reserve the right to curve the class such that your scores improve if the
final score distribution is lower than I expect.  This can only help your
grades; I will not apply a curve to reduce your score before the raw score.


\vspace{0.15in}
\noindent\textbf{Attendance}

Attendance is required in class. Part of your
grade for the semester is based upon class participation during the class
sessions. If you feel
that you have a situation that may allow for a make-up, contact the professor
immediately via email.


Excused absences are consistent with university policies in the graduate
catalog
(\url{https://catalog.ufl.edu/graduate/regulations/#text})
and require appropriate documentation.

%Note that much of the lab work in this course involves independently operating
%the observatory, which will require being present in person during non-class
%hours.

% \vskip.15in
% \noindent\textbf{Important Dates:}
% \begin{center} \begin{minipage}{3.8in}
% \begin{flushleft}
%
%
% \end{flushleft}
% \end{minipage}
% \end{center}

\vskip.15in
\noindent\textbf{Course Communication Policy:}
\begin{itemize}
\item We will use Canvas for announcements and other digital communication, so
    you are expected to regularly check Canvas.

\item We may use Slack for live communication and office hours.

\item Regular attendance on zoom or in person is essential and expected.
\end{itemize}

\vskip.15in

\noindent\textbf{Students Requiring Accommodations}

Students with disabilities requesting accommodations should first register with
the UF Disability Resource Center (352.392.8565) by providing appropriate
documentation. Once registered, students will receive an accommodation letter
which must be presented to the instructor when requesting accommodations.
Students with disabilities should follow this procedure as early as possible in
the semester

\vskip.15in
\noindent\textbf{Course Evaluation}

Students are expected to provide professional and respectful feedback on the
quality of instruction in this course by completing course evaluations online
via GatorEvals. Guidance on how to give feedback in a professional and
respectful manner is available at \url{https://gatorevals.aa.ufl.edu/students/}.
Students will be notified when the evaluation period opens, and can complete
evaluations through the email they receive from GatorEvals, in their Canvas
course menu under GatorEvals, or via \url{https://ufl.bluera.com/ufl/}. Summaries of
course evaluation results are available to students at
\url{https://gatorevals.aa.ufl.edu/public-results/}.




\vskip.15in
\noindent\textbf{Health absence / COVID policies}

We will have face-to-face instructional sessions to accomplish the student
learning objectives of this course.


\begin{itemize}
\item	You are required to wear approved face coverings at all times during
    class and within buildings.  This requirement may change over the semester.
\item	If you are experiencing any symptoms of respiratory disease (cold, flu,
    covid), please do not attend class.
\item	If you are sick, course materials will be provided to you with an
    excused absence, and you will be given a reasonable amount of time to make
    up work. Find more information in the university attendance policies.
\end{itemize}


\vskip.15in
\noindent\textbf{Online Teaching Policy}

Our class sessions may be audio visually recorded for students in the class to
refer back and for enrolled students who are unable to attend live. Students
who participate with their camera engaged or utilize a profile image are
agreeing to have their video or image recorded.  If you are unwilling to
consent to have your profile or video image recorded, be sure to keep your
camera off and do not use a profile image. Likewise, students who un-mute
during class and participate orally are agreeing to have their voices recorded.
If you are not willing to consent to have your voice recorded during class, you
will need to keep your mute button activated and communicate exclusively using
the ``chat" feature, which allows students to type questions and comments live.
The chat will not be shared. As in all courses, unauthorized recording and
unauthorized sharing of recorded materials is prohibited.


Students are requested, but not required, to keep their video on during Zoom meetings.
During breakout sessions and interactive work sessions held on zoom, both audio
and video participation will be required.  Students must have a functional webcam
and microphone.


\vskip.15in
\noindent\textbf{Class Demeanor (in person)}

Students are expected to arrive to class on time and behave in a manner that is
respectful to the instructor and to fellow students. Please avoid the use of
cell phones and restrict eating to outside of the classroom. Opinions held by
other students should be respected in discussion, and conversations that do not
contribute to the discussion should be held at minimum, if at all.

\nopagebreak
\vskip.15in
\nopagebreak
\noindent\textbf{Materials and Supplies Fees}
\nopagebreak

There are no additional fees for this course.

\vskip.15in
\noindent\textbf{University Honesty Policy}
\nopagebreak

UF students are bound by The Honor Pledge which states, `\textit{We, the members of the
University of Florida community, pledge to hold ourselves and our peers to the
highest standards of honesty and integrity by abiding by the Student Honor
Code. On all work submitted for credit by Students at the University of
Florida, the following pledge is either required or implied: “On my honor, I
have neither given nor received unauthorized aid in doing this assignment.”}'
The Honor Code
(\url{https://sccr.dso.ufl.edu/policies/student-honor-code-student-conduct-code/})
specifies a number of behaviors that are in violation of this code and the
possible sanctions. Furthermore, you are obligated to report any condition that
facilitates academic misconduct to appropriate personnel. If you have any
questions or concerns, please consult with the instructor or TA in this class.

\vskip.15in
\noindent\textbf{Counseling and Wellness Center}

Contact information for the Counseling and Wellness Center:
\url{http://www.counseling.ufl.edu/cwc/Default.aspx}, 392-1575; and the University
Police Department: 392-1111 or 9-1-1 for emergencies.



\vskip.15in
\noindent\textbf{Homework Schedule)}

Submit via Canvas
\begin{enumerate}
    \item
\end{enumerate}



\vskip.15in
\noindent\textbf{Preliminary Schedule (subject to change - I don't yet know how long these topics will take)}

Dates are Monday of the week; we meet M/W/F

\begin{itemize}
    \item Week 1 (Jan 5, 7): Syllabus, meta-discussion, gravitational collapse (Krumholz Ch 6) \\
        Problem Set 1
    \item Week 2 (Jan 10, 12, 14): Gravitational Collapse: Krumholz Ch 6, Megeath lecture 7 \\
        Problem Set 1
    \item Week 3 (Jan 19, 21): Protostar Formation: Krumholz ch 16, 17 \\
        Problem Set 1
    \item Week 4 (Jan 24, 26, 28):  Protostar Formation: Krumholz ch 16, 17 \\
        Problem Set 1 Due \\
        Problem Set 2
    \item Week 5 (Jan 31, Feb 2, 4): Accretion, Core temperature structure \\
    \item Week 6 (Feb 7, 9, 11): Asymmetry, filaments \\
    \item Week 7 (Feb 14, 16, 18): Protostellar evolution tracks \\
        Problem Set 2 due, Problem Set 3
    \item Week 8 (Feb 21, 23, 25): Flat disks \\
    \item Week 9 (Feb 28, Mar 2, 4): Flared disks \\
        Problem set 3 due (Mar 4)
    \item Spring break.
        Problem set 4 
    \item Week 10 (Mar 14, 16, 18): Outflows \\
    \item Week 11 (Mar 21, 23, 25): High-mass star formation \\
    \item Week 12 (Mar 28, 30, Apr 1):  High-mass star formation \\
        Apr 1: ArXiV presentation day \\
        Problem set 5 \\
    \item Week 13 (Apr 4, 6, 8): The IMF \\
    \item Week 14 (Apr 11, 13, 15): Clustering \\
    \item Week 15 (Apr 18, 20):  Binaries\\
    \item Week 16 (Finals / Problem Set 5 due)
\end{itemize}

\vskip.15in
\noindent\textbf{General topics covered}

Clouds to Stars:
\begin{itemize}
    \item Pressure support and spherical collapse \\
        {\it 'Til I collapse / Eminem} \\
        Problem 1: the Bonnor-Ebert sphere
    \item The Hayashi track, Kelvin-Helmholtz contraction \\
        {\it The Sun Is A Mass Of Incandescent Gas / They Might Be Giants}\\
    \item Bondi-Hoyle accretion \\
        {\it Supermassive black hole / Muse}
    \item Turbulence
        {\it All mixed up / 311}
    \item The initial mass function
    \item Star clustering
\end{itemize}

Disks:
\begin{itemize}
    \item The Toomre instability \\
        {\it You Spin Me Round (Like a Record) / Dead or Alive}
    \item Dust growth, protoplanet formation \\
        {\it When worlds collide / Powerman 5000}
    \item 
\end{itemize}

Feedback:
\begin{itemize}
    \item Accretion and outflow shocks \\
        {\it Mindfields / Prodigy}
    \item Photoionization, HII regions \\
\end{itemize}


\vskip.15in
\noindent\textbf{Additional Presentation topics}:

\begin{itemize}
    \item Collisional formation of massive stars % [https://ui.adsabs.harvard.edu/abs/2011MNRAS.413.1810B/abstract -> ?]
    \item Fluid instabilities in the ISM (Kelvin-Helmholtz, Rayleigh-Taylor)
    \item Turbulence: Any subtopic you like.  (statistical characteristics, generation, dissipation, observations)
    \item Shocks (Rankine-Hugoniot Jump Conditions)
    \item Dust from clouds to cores to disks
\end{itemize}



\vskip.15in
\noindent\textbf{Topics from Tom Megeath's class}:
\textbf{Bold} will be covered here.
\textit{Italics} are optional.

\begin{itemize}
 \item Lecture 1: Introduction to Young Stellar Objects
 \item Lecture 2: Molecular Clouds: Galactic Context and Observational Methods
 \item Lecture 3: Molecular Cloud: Properties and Evolution
 \item Lecture 4: Molecular Cloud: Turbulence and Magnetic Fields
 \item Lecture 5: \textbf{Dense Cores: Observations}
 \item Lecture 6: \textbf{Isothermal and Bonner Ebert Spheres}
 \item Lecture 7: \textbf{The Collapse of Cores and Infall}
 \item Lecture 8: \textbf{Protostars and the Collapse of Rotating Cores}
 \item Lecture 9: \textbf{The Spectral Energy Distributions of Protostars and Disks}
 \item Lecture 10: \textbf{The Spectral Energy Distributions of Disks}
 \item Lecture 11: \textit{The Evolution of Disks}
 \item Lecture 12: \textbf{The Initial Mass Function}
 \item Lecture 13: \textbf{Clusters and Associations}
 \item Lecture 14: \textit{Viscous Accretion Disks}
 \item Lecture 15: \textit{Magnetospheric Accretion}
 \item Lecture 16: \textit{Outflows}
 \item Lecture 17: \textbf{High Mass Star Formation}
 \item Lecture 18: \textbf{Pre-main Sequence Stars}
 \item Lecture 19: \textit{The Stellar Birthline}
 \item Lecture 20: \textit{Deuterium and Hydrogen Burning}
 \item Lecture 21: Main Sequence Evolution and Leaving the Main Sequence
 \item Lecture 22: Why Stars Become Red Giants
 \item Lecture 23: The Helium Flash and Horizontal Branch
 \item Lecture 24: AGB Stars and Massive Star Evolution
 \item Lecture 25: Pulsating Stars, Cepheids \& RR Lyrae stars
 \item Lecture 26: Nucleosynthesis I
 \item Lecture 27: Nucleosynthesis II
 \item Lecture 28: From AGB stars to Planetary Nebulae
 \item Lecture 29: The End Stages of Massive Stars and Supernovae
\end{itemize}


\vskip.15in
\noindent\textbf{Topics from Mark Krumholz's ``Notes on Star Formation''}:
\textbf{Bold} will be covered here.
\textit{Italics} are optional.

\begin{itemize}
 \item 1 Observing the Cold Interstellar Medium 
 \item 2 \textbf{Observing Young Stars}
 \item 3 Chemistry and Thermodynamics 
 \item 4 \textbf{Gas Flows and Turbulence}
 \item 5 Magnetic Fields and Magnetized Turbulence 
 \item 6 \textbf{Gravitational Instability and Collapse }
 \item 7 Stellar Feedback 
 \item 8 Giant Molecular Clouds 
 \item 9 \textit{The Star Formation Rate at Galactic Scales: Observations }
 \item 10 \textit{The Star Formation Rate at Galactic Scales: Theory }
 \item 11 \textbf{Stellar Clustering }
 \item 12 \textbf{The Initial Mass Function: Observations }
 \item 13 \textbf{The Initial Mass Function: Theory }
 \item 14 \textbf{Protostellar Disks and Outflows: Observations }
 \item 15 \textbf{Protostellar Disks and Outflows: Theory }
 \item 16 \textbf{Protostar Formation }
 \item 17 \textbf{Protostellar Evolution }
 \item 18 \textbf{Massive Star Formation }
 \item 19 \textit{The First Stars }
 \item 20 \textit{Late-Stage Stars and Disks }
 \item 21 \textit{The Transition to Planet Formation }
\end{itemize}



\vskip.15in
\noindent\textbf{Topics from PPVII reviews}:
\textbf{Bold} will be covered here.
\textit{Italics} are optional.

However, these reviews are not yet available!

\begin{itemize}
 \item The Life and Times of Giant Molecular Clouds	
 \item \textbf{The Solar Neighborhood in the Age of Gaia}
 \item \textbf{Star formation in the Central Molecular Zone of the Milky Way}
 \item \textbf{OB Associations}
 \item \textbf{Initial Conditions for Star Formation: A Physical Description of the Filamentary ISM}
 \item \textit{Magnetic Fields in Star Formation: From Clouds to Cores}
 \item \textbf{From Bubbles and Filaments to Cores and Disks: Gas Gathering and Growth of Structure Leading to the Formation of Solar Systems}
 \item \textbf{The Origin and Evolution of Multiple Star Systems}
 \item \textbf{A Revised Paradigm of the Role of Magnetic Fields for Disk Formation and Outflow Driving towards an Understanding of the First Stage of Planet Formation}
 \item \textbf{Accretion Variability as a Guide to Stellar Mass Assembly}
 \item \textit{Organic Chemistry in the First Phases of Solar-Like Protostars}
 \item \textit{A Theoretical Perspective of Structured Distribution of the Gas and Solids in Protoplanetary Disks} [Jaehan's review]
 \item Hydro-, Magnetohydro-, and Dust-Gas Dynamics of Protoplanetary Disks
 \item Setting the Stage for Planet Formation: Measurements and Implications of the Fundamental Disk Properties
 \item \textit{Demographics of Young Stars and Their Protoplanetary Disks: Lessons Learned on Disk Evolution and Its Connection to Planet Formation}
 \item \textit{The Role of Disk Winds in the Evolution and Dispersal of Protoplanetary Disks}
 \item Near-Infrared View of Planet-Forming Disks and Protoplanets
 \item Kinematic Structures in Planet-Forming Disks
 \item Planet-Disk Interactions and Orbital Evolution
 \item \textit{Planet Formation Theory in the Era of ALMA and Kepler: From Pebbles to Exoplanets}
 \item Short-Lived Radionuclides in Meteorites and the Sun’s Birth Environment
 \item Direct Imaging and Spectroscopy of Extrasolar Planets
 \item Exoplanet Science from the Kepler mission
 \item Architectures of Compact Multi-Planet Systems
 \item Geophysical Evolution During Rocky Planet Formation
 \item Giant Planets from the Inside-Out
 \item Exploration-Based Reconstruction of Planetesimals
 \item Chemical Habitability: Supply and Retention of Life’s Essential Elements During Planet Formation
 \item The Isotopic Link from the Planet Forming Region to the Solar System
\end{itemize}


%%%%%% THE END
\end{document}
