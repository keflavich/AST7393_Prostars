\documentclass{article}
\usepackage[inner=2.5cm,outer=2.5cm,top=4.5cm,bottom=4.5cm]{geometry}
%\documentclass[]{tufte-book} % Use the tufte-book class which in turn uses the tufte-common class
\pdfoutput=1
\usepackage[super,numbers]{natbib}
\usepackage{aas_macros}

\usepackage{microtype} % Improves character and word spacing

\usepackage{lipsum} % Inserts dummy text

\usepackage{booktabs} % Better horizontal rules in tables
\usepackage{epsdice}
\usepackage{url}
\usepackage[svgnames]{xcolor}
\usepackage[colorlinks,backref=page]{hyperref}
\hypersetup{
    colorlinks = true,
    citecolor = blue,
    linkcolor = blue,
    urlcolor = CornflowerBlue,%svgname when using xcolor
}
\urlstyle{same}

\usepackage{graphicx} % Needed to insert images into the document
\graphicspath{{graphics/}} % Sets the default location of pictures
\setkeys{Gin}{width=\linewidth,totalheight=\textheight,keepaspectratio} % Improves figure scaling

\usepackage{fancyvrb} % Allows customization of verbatim environments
\fvset{fontsize=\normalsize} % The font size of all verbatim text can be changed here

\newcommand{\hangp}[1]{\makebox[0pt][r]{(}#1\makebox[0pt][l]{)}} % New command to create parentheses around text in tables which take up no horizontal space - this improves column spacing
\newcommand{\hangstar}{\makebox[0pt][l]{*}} % New command to create asterisks in tables which take up no horizontal space - this improves column spacing

\usepackage{xspace} % Used for printing a trailing space better than using a tilde (~) using the \xspace command
\newcommand{\veco}{\ensuremath{\mathbf{\Omega}}\xspace}
\newcommand{\vecv}{\mathbf{v}\xspace}
\newcommand{\msun}{\ensuremath{\mathrm{M}_\odot}\xspace}
\newcommand{\lsun}{\ensuremath{\mathrm{L}_\odot}\xspace}
\newcommand{\ehat}{\hat{\mathbf{e}}\xspace}
\usepackage{amsmath}
\usepackage{mathtools}


\begin{document}

\noindent\textbf{Problem Set 1: Molecular Clouds}

\begin{enumerate}
    \item \textbf{Cloud Lifetimes}\\
``Typical'' molecular clouds have mean densities of $\sim100$ H$_2$ molecules per
cubic centimeter and span scales $\sim10$ pc.


Molecular clouds have temperature $T\sim20$ K.
The sound speed in molecular gas is $c_s = \sqrt{\frac{k_B T}{m}}$, where $m$ is the
mass of the particle, $k_B$ is the Boltzmann constant, and $T$ is the
temperature in Kelvin.

\begin{enumerate}
    \item The mass fraction of hydrogen, helium, and metals is\citep{Kauffmann2008}
        \begin{itemize}
            \item $X(H)=\frac{M(H)}{\mathcal{M}}=\frac{\mu_H N(H)}{\mathcal{M}}=0.71$
            \item $Y(He)=\frac{M(He)}{\mathcal{M}}=0.27$
            \item $Z(metals)=\frac{M(Z)}{\mathcal{M}}=0.02$
        \end{itemize}
        where $\mathcal{M}$ is the total mass, $\mu_H$ is the mass of a
        hydrogen atom, $N(H)$ is the number of hydrogen atoms, and $M(...)$ is the mass of the specified particle type.
        Show that the mean molecular mass, i.e., the mean mass per H$_2$
        particle, is 2.82 AMU, and the mean mass per free particle is 2.36 AMU.
        You may assume any reasonable value for $M(Z)$.
    \item Compute: What is the free-fall timescale in molecular clouds?  
        Recall, as shown in Krumholz \S6.3,
        $$t_{ff} = \sqrt{\frac{3 \pi}{32 G \rho}}$$
    \item Compute: What is the crossing time of a typical molecular cloud?  The
        crossing time is the time required for a sound wave to cross the cloud.
    \item How do these compare?
    \item If the molecular cloud forms stars at 100\% efficiency (all of the
        gas becomes stars at some point), and it collapses in one free-fall
        time, what is the star formation rate?
        Given that the Galactic molecular gas mass is $\sim10^9$ \msun, what is the MW SFR?
        Do we expect stars to form at this rate?
\end{enumerate}

\item  \textbf{Observing Clouds} \\
    Molecular clouds are primarily comprised of hydrogen molecules, H$_2$.
    Answer the following questions about the Williams Ch 7 handout
    and Ch.1 of Krumholz.

    \begin{enumerate}
        \item How do we observe molecular clouds?
            Why do we not directly observe the H$_2$ molecule?
        \item What molecules are commonly observed in the ISM?
            For each molecule, note whether it is particularly important for any specific type
            of measurement (e.g., does it trace `high'- or `low'- density gas?  Can it be
            used to measure temperature?).\footnote{
            You should not provide an exhaustive list, just note the molecules discussed in the
            assigned chapters.  Fully exhaustive lists can be found at
            \url{http://www.astrochymist.org/astrochymist_ism.html} and
            \url{https://ui.adsabs.harvard.edu/abs/2022ApJS..259...30M/abstract}.}
        \item (G) In a 10K molecular cloud, what is the relative population in the J=1 and J=3 states of CO?
            Assume LTE conditions, i.e., $n>>n_{crit}$.
            % n_1/n_2 = g_1/g_2 * e^(-E1/T)/e^(-E2/T)
            % g = 2J + 1
            % 3 * np.exp(-5.5/10) / (7 * np.exp(-33.2/10)) = 6.8
            % J=1 to J=3 ratio = 6.8, inverse is 0.14
        \item (G) What is this value for a cloud with density $n(H_2)=10^2 \mathrm{cm}^{-3}$?
            % n_crit = 3.5e4
            % eqn 1.16 of Krumholz
            % value of upper/lower is decreased by 1/(1+ncrit/n) = 0.14 / (1 + 3.5e4/1e2) = 0.0004
            % inverse is ~2500
        \item (G) Should CO 3-2 be detectable in a typical molecular cloud?  Why or why not?
    \end{enumerate}

\item \textbf{Molecular Cloud Kinematics} \\

    Larson's Laws include the size-linewidth relation, $\sigma \propto L^{0.5}$, where $\sigma$ is the line width and $L$ is the size.
    It was first observed, and still is most often referenced, in emission lines of CO.
    In the questions below, be sure to consider what particles are being discussed.

    \begin{enumerate}
        \item (G) Typical clouds in the Solar Neighborhood have line widths $FWHM\approx 1.5 \mathrm{km~s}^{-1}$ on $\sim1$ pc size scales.
            At what size scale does the velocity dispersion predicted by this relation equal the sound speed in a 20K molecular cloud?
            % answer: fwhm -> sigma, (size / 1 pc)^0.5 = c_s / sigma(1pc) = 0.26 km/s / 0.7 km/s
            % results in 0.14 pc
            % 0.7 km/s comes from Shetty+ 2012 figure 8
            % 0.26 km/s = sqrt(k_B * T / 2.4 AMU)
            %
            % if we adopt 28 AMU for CO, the speed is 0.077 km/s and the size scale is 0.012 pc
        \item (G) For an average cloud density $n_{H_2}\approx10^2$ cm$^{-3}$, what magnetic field is required for $v_{a} > c_s$?
            % gauss = u.cm**-0.5 * u.g**0.5 *u.s**-1
            % (((constants.k_B * 20 * u.K / (2.4*u.Da))**0.5).to(u.km/u.s)  * (4*np.pi*(100*2.4*u.Da*u.cm**-3))**0.5).to(gauss)
            % 1.8 microGauss
        \item (G) For a cloud with integrated intensity in CO 1-0 of 100 K km s$^{-1}$ over an area of $r\approx1^{\circ}$
            at a distance of 200 pc, what is its mass?
            %  Xco = 2x10^{20} cm^-2/(K km/s)^-1
            %  N(H2) = 2e22 cm^-2
            %  ((2e22*u.cm**-2) * (1*u.deg * 200*u.pc).to(u.pc, u.dimensionless_angles())**2 * np.pi * 2.8*u.Da).to(u.M_sun)
            %  1.7e4 Msun

    \end{enumerate}

%\item \textbf{Stars and Gas} \\
%    Why do galaxies appear roughly the same when you look at the gas vs. when you look at the stars?

\end{enumerate}


\bibliographystyle{aasjournal}
\bibliography{bibliography}
\end{document}


