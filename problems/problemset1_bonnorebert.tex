%\documentclass[twoside]{tufte-book} % Use the tufte-book class which in turn uses the tufte-common class
\documentclass{article}
\pdfoutput=1
\usepackage{url}
% \usepackage{natbib}
% 
% \usepackage{microtype} % Improves character and word spacing
% 
% \usepackage{lipsum} % Inserts dummy text
% 
% \usepackage{booktabs} % Better horizontal rules in tables
% 
% \usepackage{graphicx} % Needed to insert images into the document
% \graphicspath{{graphics/}} % Sets the default location of pictures
% \setkeys{Gin}{width=\linewidth,totalheight=\textheight,keepaspectratio} % Improves figure scaling
% 
% \usepackage{fancyvrb} % Allows customization of verbatim environments
% \fvset{fontsize=\normalsize} % The font size of all verbatim text can be changed here
% 
% \newcommand{\hangp}[1]{\makebox[0pt][r]{(}#1\makebox[0pt][l]{)}} % New command to create parentheses around text in tables which take up no horizontal space - this improves column spacing
% \newcommand{\hangstar}{\makebox[0pt][l]{*}} % New command to create asterisks in tables which take up no horizontal space - this improves column spacing

\usepackage{xspace} % Used for printing a trailing space better than using a tilde (~) using the \xspace command

\begin{document}

\noindent\textbf{Problem Set 1: The Bonnor-Ebert Sphere}

This problem set is based on Krumholz chapter 6
(\url{https://github.com/Open-Astrophysics-Bookshelf/star_formation_notes})
with additions and clarifications by Adam Ginsburg.

See also
\url{http://astro1.physics.utoledo.edu/~megeath/ph6820/lecture6_ph6820.pdf} and
Palla \& Stahler ch 9.1,
which walk through the same derivation.   
\\

%\textbf{The Bonnor-Ebert Sphere.}\\
Here we will investigate the properties of hydrostatic spheres of gas supported
by thermal pressure. These are reasonable models for thermally-supported
molecular cloud cores. Consider an isothermal, spherically-symmetric cloud of
gas with mass $M$ and sound speed $c_s$, confined by some external pressure
$P_{\mathrm{s}}$ on its surface.

\begin{enumerate}
\item For the moment, assume that the gas density inside the sphere is uniform.
    \textbf{Use the virial theorem to derive a relationship between $P_{\mathrm{s}}$
    and the cloud radius $R$. }

    Recall:
    $$\frac{1}{2}\ddot{I} = 2 (\mathcal{T} - \mathcal{T}_S) + \mathcal{W}$$
    $\mathcal{T}_S$ is the average surface pressure, given by
    $$\mathcal{T}_S = \int_S r P dS = r P \int_S dS = 4 \pi R^2 R P_S = 4 \pi R^3 P_S$$
    assuming a sphere with a constant surface pressure term. \\
    $\mathcal{W}$ is the gravitational potential
    $$\mathcal{W} = - a \frac{GM^2}{R}$$

\item Show that there is a maximum surface pressure $P_{\mathrm{s,max}}$ for
    which virial equilibrium is possible and derive its value.
    (find an extremum of $P_S(R)$ with respect to $R$)


\item Now we will compute the true density structure. Consider first the equation of hydrostatic balance,
\begin{displaymath}
-\frac{1}{\rho}\frac{d}{dr} P = \frac{d}{dr} \phi,
\end{displaymath}
where $P = \rho c_s^2$ is the pressure and $\phi$ is the gravitational
potential. Let $\rho_c$ be the density at $r=0$, and choose a solution such that
$\phi = 0$ at $r=0$. Integrate the equation of hydrostatic balance to obtain an
expression relating $\rho$, $\rho_c$, and $\phi$.

(recall: $\frac{d }{dx}\left(\ln y(x)\right) = \frac{1}{y} \frac{d y}{dx}$)

\item Now consider the Poisson equation for the potential,
\begin{displaymath}
\frac{1}{r^2}\frac{d}{dr}\left(r^2 \frac{d\phi}{dr}\right) = 4 \pi G \rho.
\end{displaymath}
Use your result from the previous part to eliminate $\rho$, and define $\psi \equiv \phi/c_s^2$. Show that the resulting equation can be non-dimensionalized to give the isothermal Lane-Emden equation:
\begin{displaymath}
\frac{1}{\xi^2}\frac{d}{d\xi}\left(\xi^2 \frac{d\psi}{d\xi}\right) = e^{-\psi}.
\end{displaymath}
where $\xi = r/r_0$\footnote{$\xi$ is \texttt{\textbackslash xi}.  It is very commonly used
for this dimensionless radius parameter, but I can't write it to save my life.
If you're doing this by hand, feel free to use a different variable name.}.
What value of $r_0$ is required to obtain this equation?
\label{item:r0}


\item Numerically integrate the isothermal Lane-Emden equation subject to the
    boundary conditions $\psi=d\psi/d\xi = 0$ at $\xi=0$; the first of these
    conditions follows from the definition of $\psi$, and the second is
    required for the solution to be non-singular. From your numerical solution,
    plot both $\psi$ and the density contrast $\rho/\rho_c = e^{-\psi}$ versus
    $\xi$.

    \begin{enumerate}
        \item Break the second-order differential equation into a system of first-order ODEs using the definition
    $$ \psi ' = \frac{d\psi}{d\xi} $$
        \item Solve for the boundary conditions for $r\sim0$ by producing an expansion around $r=0$.
            Retain first-order terms for $\frac{d \psi}{d\xi}$.
            $$ \psi(\xi) = a_0 + a_1 \xi + a_2 \xi^2 + a_3 \xi^3 + \mathcal{O}(4)$$
    \end{enumerate}

    scipy's ODE solver can be used like so:
\begin{verbatim}
def derivatives(independent_variable, dependent_variable):
    return [first_derivative, second_derivative]

sol = solve_ivp(derivatives,
                t_span=(integral_start, integral_stop),
                y0=[y0, yprime0],
               )
\end{verbatim}

For further details about the input and output, look at the scipy documentation.



\item The total mass enclosed out to a radius $R$ is
\begin{displaymath}
M = 4\pi \int_0^R \rho r^2 \, dr.
\end{displaymath}
Show that this is equivalent to
\begin{displaymath}
M =\frac{c_s^4}{\sqrt{4\pi G^3 P_s}} \left(e^{-\psi/2}\xi^2 \frac{d\psi}{d\xi}\right)_{\xi_s},
\end{displaymath}
where
\begin{eqnarray*}
\xi_s & \equiv & \frac{R}{r_0} \\
P_s & \equiv & \rho_s c_s^2.
\end{eqnarray*}
$P_s$ is the pressure at the surface, and the pressure contrast is the same as the density contrast $$\frac{P}{P_c} = \frac{\rho}{\rho_c} = e^{-\psi}$$
Hint: to evaluate the integral, it is helpful to use the isothermal Lane-Emden equation to substitute.


\item Plot the dimensionless mass $m = M/(c_s^4/\sqrt{G^3 P_s})$ versus the dimensionless density contrast $\rho_s/\rho_c=e^{-\psi_s}$, where $\psi_s$ is the value of $\psi$ at $\xi=\xi_s$. You will see that $m$ reaches a finite maximum value $m_{\mathrm{max}}$ at a particular value of $\rho_c/\rho_s$. Numerically determine $m_{\mathrm{max}}$, along with the density contrast $\rho_c/\rho_s$ at which it occurs.
\item The existence of a finite maximum $m$ implies that, for a given dimensional mass $M$, there is a maximum surface pressure $P_s$ at which a cloud of that mass can be in hydrostatic equilibrium. Solve for this maximum, and compare your result to the result you obtained in part (a).
\item Conversely, for a given surface pressure $P_s$ and sound speed $c_s$ there exists a maximum mass at which the cloud can be in hydrostatic equilibrium, called the Bonnor-Ebert mass $M_{\mathrm{BE}}$. Obtain an expression for $M_{\mathrm{BE}}$ in terms of $P_s$ and $c_s$. In a typical low-mass star-forming region, the surface pressure on a core might be $P_{\mathrm{s}}/k_{\rm B} = 3\times 10^5$ K cm$^{-3}$. Compute this mass for a core with a temperature of 10 K, assuming the standard mean molecular weight $\mu=2.3$.\\
    By contrast, in Galactic centers and dense star forming regions in the early universe, the temperature and pressure may have been higher.  What is the Bonnor-Ebert mass for $P_{\mathrm{s}}/k_{\rm B} = 10^7$ K cm$^{-3}$ and T=$50$ K?
    \label{item:m_be}


\item Now that you have computed the Bonnor-Ebert mass and surface pressure, determine the size $r_0$ using the equation derived in Part \ref{item:r0}.
    You will need to determine the central density $\rho_c$ as well.  What is this value in g cm$^{-3}$ and particles per cubic centimeter?
    Evaluate these numbers for both cases (`local' and `Galactic center') in Part \ref{item:m_be}.

\item Plot the enclosed mass and the average enclosed density as a function of radius in real physical units.
    Plot these values for both cases in Part \ref{item:m_be}.

\item Identify (fit) an approximation to the Bonnor-Ebert density profile of the form:
    \begin{displaymath}
        \rho(r) = \rho_c \frac{(c \cdot r_0)^\alpha}{(c \cdot r_0)^\alpha + r^\alpha}
    \end{displaymath}
    where $c$ and $\alpha$ are constants to be determined.  
    For $\alpha=2$, this equation is called a Plummer profile.  More generally, this
    is a power-law profile with a flattened central core; i.e., for $r>>r_0$, this
    profile approaches $\rho(r) = r^{-\alpha}$.  For $r<<r_0$, $\rho=\rho_c$.

    

    Evaluate: over what range is this numerical approximation reasonable?
    (use plots in your evaluation)

 \item B68 is a real Bonnor-Ebert sphere.  It has a measured shape parameter $\xi_{max}=6.9$.
     Is the core stable?  Recall that in part 7, you determined the maximum stable dimensionless
     mass.

 \item B68 has an outer radius of 150" at its distance of 128 pc.  Its average temperature
     is $T=10.5$ K.  What is its central density $\rho_c$?  Recall Part 4 and the definition $\xi=r/r_0$ (so $\xi_{max} = r_{max}/r_0$).

 \item Compare the surface pressure of B68 to the typical ISM pressure $P/k = 2\times10^4 \mathrm{cm}^{-3} \mathrm{K}$.
     Recall from part 6 that $$\frac{P}{P_c} = \frac{\rho}{\rho_c} = e^{-\psi}$$
    and the ideal gas law ($P = \rho k_B T$).
\end{enumerate}

\end{document}
