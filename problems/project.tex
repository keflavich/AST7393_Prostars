OBSOLETE VERSION
\documentclass{article}
\pdfoutput=1
%\usepackage{xspace} % Used for printing a trailing space better than using a tilde (~) using the \xspace command

\begin{document}

\title{Project for AST 7939: Star Formation}
\noindent \textbf{\large Project for AST 7939: Star Formation}

You will do an independent project addressing one of the key topics in the
class.

The goal is to develop a deep understanding of one specific topic
and teach that to the class.
This will serve as training both for research and teaching.

You will do the following:
\begin{enumerate}
    \item Identify a topic relevant to the course and discuss it with Adam.
        We will set deadlines based on the topic chosen
    \item Find 1-2 reviews or textbook chapters on the topic from the last 30
        years, with strong preference for more recent reviews.  Do a literature
        review of more recent work, identifying important updates since the review
        was written.
    \item Prepare a presentation on the topic, summarizing the context of the
        problem, the problem, past and recent progress on the problem, and open
        questions related to the problem.
    \item Prepare a problem set related to the topic.  The problem set should
        include a quantitative demonstration of the core problem of the topic.
    \item Design an experiment, observation, or theoretical project that can
        address one of the open problems in the topic.  
\end{enumerate}



\noindent \textbf{Presentation:}

You will give a lecture on the selected topic. 

You should aim to teach the material at a level that an undergraduate major
would be able to follow.  You should try to present in a way that is accessible
to multiple learning styles; for example, interactivity in the form of audience
questions and group work is welcome.  However, you will need to cover all of
the topic material within one class period, so be aware of time limitations.


\noindent \textbf{Problem Set:}

You will design a problem set based on your selected topic.

The problem set should resemble problem sets you've seen in classes, including
problems with some simple, open-ended questions that require both insight and
creativity to answer, and some more closely guided questions that are meant to
guide students to a common solution path.


\noindent \textbf{Experiment Design:}

You will practice converting an open-ended, research question into an 
experiment or theory project.
You do not need to carry out the project, but you should outline it in
detail.
You will write up your experimental plan as an observing, computing time,
or funding proposal.

The proposal will be limited to 4 pages, including figures and references.
However, if you choose to write an observing, grant, or computing time proposal
for submission, you may use their format.

\end{document}
